\documentclass{standalone}
\usepackage{fontawesome}
\renewcommand{\familydefault}{\sfdefault}

\usepackage{tikz}
\usetikzlibrary{arrows.meta,backgrounds,calc,fit,shadows,positioning}

% document shape
\makeatletter
\pgfdeclareshape{document}{
  \inheritsavedanchors[from=rectangle] % this is nearly a rectangle
  \inheritanchorborder[from=rectangle]
  \inheritanchor[from=rectangle]{center}
  \inheritanchor[from=rectangle]{north}
  \inheritanchor[from=rectangle]{north east}
  \inheritanchor[from=rectangle]{east}
  \inheritanchor[from=rectangle]{south east}
  \inheritanchor[from=rectangle]{south}
  \inheritanchor[from=rectangle]{south west}
  \inheritanchor[from=rectangle]{west}
  \inheritanchor[from=rectangle]{north west}
  % ... and possibly more
  \backgroundpath{% this is new
  % store lower right in xa/ya and upper right in xb/yb
  \southwest \pgf@xa=\pgf@x \pgf@ya=\pgf@y
  \northeast \pgf@xb=\pgf@x \pgf@yb=\pgf@y
  % compute corner of ‘‘flipped page’’
  \pgf@xc=\pgf@xb \advance\pgf@xc by-10pt % this should be a parameter
  \pgf@yc=\pgf@yb \advance\pgf@yc by-10pt
  % construct main path
  \pgfpathmoveto{\pgfpoint{\pgf@xa}{\pgf@ya}}
  \pgfpathlineto{\pgfpoint{\pgf@xa}{\pgf@yb}}
  \pgfpathlineto{\pgfpoint{\pgf@xc}{\pgf@yb}}
  \pgfpathlineto{\pgfpoint{\pgf@xb}{\pgf@yc}}
  \pgfpathlineto{\pgfpoint{\pgf@xb}{\pgf@ya}}
  \pgfpathclose
  % add little corner
  \pgfpathmoveto{\pgfpoint{\pgf@xc}{\pgf@yb}}
  \pgfpathlineto{\pgfpoint{\pgf@xc}{\pgf@yc}}
  \pgfpathlineto{\pgfpoint{\pgf@xb}{\pgf@yc}}
  \pgfpathlineto{\pgfpoint{\pgf@xc}{\pgf@yc}}
  }
}
\makeatother

\definecolor{progblue}{RGB}{41,128,185}

\tikzstyle{document_base}=[
  align=center,
  draw=black!35,
  fill=white,
  minimum height=28.2mm,
  minimum width=20mm,
  shape=document,
  thick,
]

\tikzstyle{document}=[
  document_base,
  drop shadow,
]

\tikzstyle{arrow}=[-{Latex[length=2mm,width=2mm]},thick]

\begin{document}

% program style
\tikzstyle{program_base}=[
  align=center,
  draw=progblue,
  fill=progblue!10,
  inner sep=5mm,
  rectangle,
  rounded corners=2mm,
  thick,
]

% other styles
\tikzstyle{bgbox}=[
  rectangle,
]
\tikzstyle{levelline}=[
  dashed,
  draw=black!20,
  ultra thick,
]
\tikzstyle{row_label}=[rotate=90]

% variants
\tikzstyle{document_id}=[
  document_base,
  drop shadow,
]
\tikzstyle{program_id}=[
  program_base,
  drop shadow,
]
\tikzstyle{bgbox_id}=[
  bgbox,
]
\tikzstyle{arrow_id}=[
  arrow,
]

\begin{tikzpicture}

  \newcommand{\rowlabel}[1]{\huge #1}
  \newcommand{\fileext}[1]{\large\texttt{.#1}}

  % source level
  \node[document_id]                                      (doc_markdown)          {\fileext{md}};

  % IR level
  \node[program_id,below=of doc_markdown]                 (prog_innoconv)         {innoConv};
  \node[document_id,below=of prog_innoconv]               (doc_json)              {\fileext{json}};

  % presentation level
  \node[program_id,below left=of doc_json,anchor=north east]  (prog_webapp)           {{\Huge\faLaptop}\\Web app};
  \node[program_id,below right=of doc_json,anchor=north west] (prog_app)              {{\Huge\faMobile}\\Mobile app};
  \node[document_id,below=of doc_json]                  (doc_pdf)               {\fileext{pdf}};

  % arrows
  \path[arrow_id] % innoDoc
    (doc_markdown)    edge node (arr_north) {} (prog_innoconv)
    (prog_innoconv)   edge node             {} (doc_json)
    (doc_json)        edge node (arr_south) {} (prog_webapp)
    (doc_json)        edge node             {} (prog_app)
    (doc_json)        edge node             {} (doc_pdf);

  % level labels
  \begin{scope}
    \node[row_label] at (-55mm,    1mm) {\rowlabel{Source}};
    \node[row_label] at (-55mm,  -50mm) {\rowlabel{Intermediate}};
    \node[row_label] at (-55mm, -109mm) {\rowlabel{Presentation}};
  \end{scope}

  % background layer
  \begin{scope}[on background layer={anchor=west}]
    % bg box
    \node[
      bgbox,
      fit=(doc_markdown)(prog_webapp)(prog_app)(doc_pdf),
      inner sep=10mm,
    ] (bgbox) {};
    % level separator lines
    \draw let
      \p1 = (arr_north),
      \p2 = (arr_south)
    in
      (bgbox.west |- 0, \y1) -- (bgbox.east |- 0, \y1) [levelline]
      (bgbox.west |- 0, \y2) -- (bgbox.east |- 0, \y2) [levelline];
  \end{scope}

\end{tikzpicture}

\end{document}
